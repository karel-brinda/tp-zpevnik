% -*-coding: utf-8 -*-

\zp{Sbohem, lásko}{Waldemar Matuška}

\zs
Ať bylo <C>mně i <F>jí tak <G>šestnáct <C>let, <F> <G>

zeleným <C>údolím <Ami>jsem si ji <Dmi>ved', <G7>

<C>byla krásná, to <C7>vím, a já měl <F>strach, jak <Fmi>říct,

když na řa<C>sách slzu <G7>má velkou jako <C>hrách: <F> <C>
\ks

\zr
\uv{<C7>Sbohem, <F>lásko, nech mě <Fmi>jít, nech mě <Emi>jít, bude <Ami>klid,

žádnej <Dmi>pláč už nespra<G7>ví ty mý <C>nohy toula<C7>vý,

já tě <F>vážně měl moc <Fmi>rád, co ti <Emi>víc můžu <Ami>dát?

Nejsem <Dmi>žádnej ide<G7>ál, tak nech mě <C>jít zas <F>o dům <C>dál.}
\kr

\zs
A tak šel čas, a já se toulám dál,

v kolika údolích jsem takhle stál,

hledal slůvka, co jsou jak hojivej fáč, bůhví,

co jsem to zač, že přináším všem jenom pláč.
\ks

\zr\kr

Rec: Já nevím, kde se to v člověku bere -- ten neklid, co ho tahá z místa na 
místo, co ho nenechá, aby byl sám se sebou spokojený jako většina ostatních, 
aby se usadil, aby dělal jenom to, co se má, a říkal jenom to, co se od něj 
čeká, já prostě nemůžu zůstat na jednom místě, nemůžu, opravdu, fakt.

\zr\kr

\kp
