% -*-coding: utf-8 -*-

\zp{Hrneček}{Samson a jeho parta}

\zs
Přečti mi \Ch{G}{pohádku} z hrnečku \Ch{H#maj7}{od kafe}
a ať je \Ch{G}{v ní něco} \Ch{H#maj7}{hezkýho.}

A ať jsem \Ch{G}{v pořádku}, ať mě svět \Ch{H#maj7}{nerafe}
a ať \Ch{G}{mám hodnýho} \Ch{H#maj7}{mužskýho.}

Ať se mi \Ch{Ami}{zadaří a ať mě} \Ch{D7}{nemlátí,}
ať mi ten \Ch{Ami}{lógr už svět v koláč} \Ch{D7}{obrátí.}

Ať mi v něm \Ch{G}{dopřeje} mandle a \Ch{H#maj7}{hrozinky,}
prosím tě, \Ch{G}{uděj se,} štěstíčko \Ch{H#maj7}{malinký.}
\ks

\zr
Do dveří \Ch{Ami}{kopou, když} chtějí \Ch{D7}{za ně,}

ti s velkou \Ch{G}{stopou,} co šetří \Ch{H#maj7}{dlaně.}

Na dveře \Ch{Ami}{buší, když} chtějí \Ch{D7}{vstoupit,}

pěvci bez \Ch{G}{uší, hluční} a \Ch{H#maj7}{hloupí.}

Zaťuká \Ch{Ami}{lehce,} kdo hledá \Ch{D7}{teplo,}

zaťuká \Ch{Ami}{lehce a} \Ch{D7}{nic víc} \Ch{G}{nechce.}
\kr

\zs
Přečti mi pohádku z hrnečku od kafe
a ať je v ní něco hezkýho.

A ať jsem v pořádku, ať mě svět nerafe
a ať mám hodnýho mužskýho.

Do sítě slůvek, než mi tě střelej,
chyť se v mym těle, ušatej skvělej.

Úplně celej měsíc je kulatej,
tak něco dělej, nesmělej ušatej.
\ks

\zr \kr

\zs
Přečti mi pohádku z hrnečku od kafe
a ať je v ní něco hezkýho.

A ať jsem v pořádku, ať mě svět nerafe
a ať mám hodnýho mužskýho.

Ať se mi zadaří a ať mě nemlátí,
ať ten lógr už svět v koláč obrátí.

% mezery u repetice nejsou kvuli tomu, aby se radek nerozpadl na dva
Ať mi v něm dopřeje mandle a hrozinky,
/:prosím tě, uděj se, štěstíčko malinký.:/
\ks

\kp
