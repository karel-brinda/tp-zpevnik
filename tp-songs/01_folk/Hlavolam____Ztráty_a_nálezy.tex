% -*-coding: utf-8 -*-

\zp{Ztráty a nálezy}{Hlavolam}

\Ch{Emi}{} \Ch{D}{} \Ch{Ami}{} \Ch{Emi}{} \Ch{H}{}

\zs
\Ch{Emi}{Kolikrát} jsem minul tuhletu výlohu \Ch{D}{s panáčkem} od sazí,

\Ch{Ami}{nikdy} jsem nepřišel na to, co znamená \uv{\Ch{Emi}{ztráty} a nále\Ch{H}{zy}},

\Ch{Emi}{když} dneska pomyslím na ta dvě slovíčka, \Ch{D}{trochu} mě zamrazí,

\Ch{Ami}{vždyť} i v mém životě hrají hlavní roli \Ch{Emi}{ztráty} a nále\Ch{H}{zy.}
\ks

\zr
A tak ať \Ch{Emi}{hledám} či nehle\Ch{C}{dám,} staré \Ch{D}{ztrácím,} nové nalé\Ch{G}{zám},

něco \Ch{Emi}{mám} a pak ne\Ch{C}{mám} zase \Ch{H}{nic},

říka\Ch{Emi}{jí, že} je jen \Ch{C}{klam} úsměv \Ch{D}{čarokrásných} \Ch{G}{dam,}

stále \Ch{Emi}{ztrácím} a nalé\Ch{C}{zám}, někdy \Ch{H}{málo} a někdy víc.
\kr

\zs
Najít můžeš černé peklo a ztratit zas nejmodřejší nebe,

vždycky však pamatuj na ztráty a nálezy, nechoď hledat sebe,

až jednou uvidíš, že nemáš vůbec nic, neztrácej naději,

vždyť kdo hledá, najde, a tak znovu hledat začínej raději.
\ks

\zr
Jednou ten koloběh poznáš, že staré ztrácíš, nové nalézáš,

něco máš a hned nemáš zase nic,

proto včerejšek dnes smaž, málo vezmeš a snad více dáš,

stále ztrácíš a nalézáš, někdy málo a někdy víc.
\kr

Co dál k tomu \Ch{Emi}{říct...} \Ch{D}{} \Ch{Ami}{} \Ch{Emi}{} \Ch{H}{} \Ch{Emi}{}

\kp
























