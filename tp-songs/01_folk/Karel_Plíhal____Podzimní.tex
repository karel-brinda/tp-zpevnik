% -*-coding: utf-8 -*-

\zp{Podzimní}{Karel Plíhal}

\zs
\Ch{A}{Podzimní} obloha dala se \Ch{D}{do gala,}

\Ch{E}{večerní} vánek se do vlasů \Ch{A}{vplétá.}

A po tý obloze na křídle \Ch{D}{rogala}

s \Ch{E}{tím vánkem} ve vlasech Markéta \Ch{A}{létá.}

Nebe je modrý jako mý \Ch{D}{džíny,}

\Ch{E}{tak jsme si} zpívali s klukama \Ch{A}{za mlada.}

Zmizely smutky a podzimní \Ch{D}{splíny,}

\Ch{E}{prostě} to všechno, co Markéta \Ch{A}{nerada.}
\ks

\zs
Vysoko na nebe, hluboko do polí

Markéta létá a přitom si zpívá.

Co oči nevidí, to srdce nebolí,

je totiž podzim a brzo se stmívá.

Zmizely splíny a přívaly pláče

a s nima ty protivný přízraky z minula.

Připravte obvazy, dlahy a fáče,

kdyby se náhodou se zemí minula.
\ks

\zs
= 1.
\ks

\zs
= 2.
\ks

4× /: Podzimní obloha dala se do gala,

večerní vánek se do vlasů vplétá. :/

\kp
