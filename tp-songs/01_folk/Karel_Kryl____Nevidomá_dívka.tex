% -*-coding: utf-8 -*-

\zp{Nevidomá dívka}{Karel Kryl}

\zs
V \Ch{C}{zahradě} \Ch{Dmi}{za cihlovou} \Ch{C}{zídkou,} \Ch{Dmi}{}

\Ch{C}{popsanou} v \Ch{Dmi}{slavných výro}\Ch{C}{čích,} \Ch{Dmi}{}

\Ch{C}{sedává} \Ch{Dmi}{na podzim} \Ch{E}{na trávě} \Ch{Ami}{před besídkou}

\Ch{F}{děvčátko} s \Ch{G}{páskou} na o\Ch{E}{čích.}
\ks

\zs
Pohádku o mluvícím ptáku nechá si přečíst z notesu,

pak pošle polibek po chmýří na bodláku na vymyšlenou adresu.
\ks

\zr
Prosím vás, \Ch{Ami}{nechte ji,} ach, \Ch{Dmi}{nechte ji,}

\Ch{Ami}{tu nevidomou} \Ch{G}{dívku,}

\Ch{Dmi}{prosím vás,} \Ch{G}{nechte ji si} \Ch{E}{hrát,}

\Ch{Dmi}{vždyť možná} hraje si \Ch{Ami}{na slunce} s nebesy,

\Ch{F}{jež nikdy} neuvidí, \Ch{G}{ač} ji bude \Ch{E}{hřát.}
\kr

\zs
Pohádku o mluvícím ptáku a o třech zlatých jabloních

a taky o lásce, již v černých květech máku přivezou jezdci na koních.
\ks

\zs
Pohádku o kouzelném slůvku, jež vzbudí všechny zakleté,

pohádku o duze, jež spává na ostrůvku, na kterém poklad najdete.
\ks

\zr\kr

\zs
Rec: Rukama dotýká se květů, a neruší ji motýli,

jen trochu hraje si s řetízkem amuletu, jen na chvíli, jen na chvíli.
\ks

\zr\kr

\kp
