% -*-coding: utf-8 -*-

\zp{Velrybářská výprava}{Pacifik}

\zs
Jed\Ch{E}{nou} plác mě přes \Ch{A}{rameno} Joh\Ch{H7}{ny} zvanej Knecht: \Ch{E}{}

\uv{Mám \Ch{C#mi}{pro tebe}, ho\Ch{F#mi}{chu,} v pácu moc \Ch{H7}{fajnovej} kšeft!} \Ch{E}{}

Objednal hned litr \Ch{A}{rumu} a \Ch{H7}{pak} nežně řval: \Ch{E}{}

\uv{Sbal, \Ch{C#mi}{se je}dem na \Ch{F#mi}{velryby}, prej \Ch{H7}{až} za polár.} \Ch{E}{}
\ks

\zr
Výprava vel\Ch{E}{rybářská} kole\Ch{A}{m} Grónska nez\Ch{H7}{dařila} se,\Ch{E}{}

protože nejeli \Ch{C#mi}{jsme} na vel\Ch{F#mi}{ryby}, ale \Ch{H7}{na} mrože. \Ch{E}{}
\kr

\zs
Briga zvaná Malá Kitty kotví v zátoce,
nakládaj se sudy s rumem, maso, ovoce,
vypluli jsme časne zrána, směr severní pól,
dřív, než přístav zmizel z očí, každej byl namol.
\ks

\zr  \kr

\zs
Na loď padla jinovatka, s ní třeskutej mráz,
hoň velryby v kupách ledu, to ti zlomí vaz,
na pobreží místo ženskejch mávaj tučňáci,
v tomhle kraji beztak nemáš jinou legraci.
\ks

\zr  \kr

\zs
Když jsme domů připluli, už psal se přítí rok,
starej rejdař povídá, že nedá ani flok:
"Místo velryb v Grónským moři zajímal vás grog,
tuhle práci zaplatil by asi jenom cvok."
\ks

\zr  \kr

\zs
Tohleto nám neměl říkat, teď to dobře ví,
stáhli jsme mu kuži z těla, tomu hadovi,
z paluby pak slanej vítr jeho tělo smet, chachacha,
máme velryb plný zuby, na to vezmi jed.
\ks

\zr  \kr

\kp






















