% -*-coding: utf-8 -*-

\zp{Fernande}{Georges Brassens}

\zs
<C>Une manie de vieux garçon,
moi <F>j'ai pris l'habi<E>tude

<G>D'agré<A7>menter ma<Dmi> soli<E>tude,
aux <D7>accents <G7>de cette chan<C>son <D7>
\ks

\zr 
Quand <G>je pense à Fer<Ami>nande,
je <D7>bande, je <G>bande

Quand j'pense à Féli<C>cie,
je bande <G>aussi

Quand j'pense à Lé<C>onore,
mon <D7>dieu je bande en<G>core

Mais <H7>quand j'pense à Lu<Emi>lu,
<Ami>là <D7>je ne bande <E>plus

La <H7>bandaison pa<Emi>pa,
ça n'se com<H7>man<D7>de <G>pas. ~~~ <C>
\kr

\zs
C'est cette mâle ritournelle,
cette antienne virile 

Qui retentit dans la guérite,
de la vaillante sentinelle.
\ks

\zr \kr

\zs
Afin de tromper son cafard,
de voir la vie moins terne 

Tout en veillant sur sa lanterne,
chante ainsi le gardien de phare.
\ks

\zr \kr

\zs
Après la prière du soir,
comme il est un peu triste 

Chante ainsi le séminariste,
à genoux sur son reposoir.
\ks

\zr \kr

\zs
A l'étoile où j'était venu,
pour ranimer la flamme 

J'entendis émus jusqu'au larmes,
la voix du soldat inconnu.
\ks

\zr \kr

\zs
Et je vais mettre un point final,
à ce chant salutaire 

En suggérant au solitaire,
d'en faire un hymne national.
\ks

\zr \kr

\kp
