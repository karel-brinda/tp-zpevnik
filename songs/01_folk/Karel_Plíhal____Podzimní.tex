% -*-coding: utf-8 -*-

\zp{Podzimní}{Karel Plíhal}

\zs
<A>Podzimní obloha dala se <D>do gala,

<E>večerní vánek se do vlasů <A>vplétá.

A po tý obloze na křídle <D>rogala

s <E>tím vánkem ve vlasech Markéta <A>létá.

Nebe je modrý jako mý <D>džíny,

<E>tak jsme si zpívali s klukama <A>za mlada.

Zmizely smutky a podzimní <D>splíny,

<E>prostě to všechno, co Markéta <A>nerada.
\ks

\zs
Vysoko na nebe, hluboko do polí

Markéta létá a přitom si zpívá.

Co oči nevidí, to srdce nebolí,

je totiž podzim a brzo se stmívá.

Zmizely splíny a přívaly pláče

a s nima ty protivný přízraky z minula.

Připravte obvazy, dlahy a fáče,

kdyby se náhodou se zemí minula.
\ks

\zs
= 1.
\ks

\zs
= 2.
\ks

4× /: Podzimní obloha dala se do gala,

večerní vánek se do vlasů vplétá. :/

\kp
