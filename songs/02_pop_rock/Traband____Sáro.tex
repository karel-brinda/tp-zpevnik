% -*-coding: utf-8 -*-

\zp{Sáro}{Traband}

\zr
<Ami>Sáro, <Emi>Sáro, v <F>noci se mi <C>zdálo,
že <F>tři andělé <C>boží k nám <F>přišli na o<G>běd.

<Ami>Sáro, <Emi>Sáro, jak <F>moc a nebo <C>málo
mi <F>chybí, abych <C>tvojí duši <F>mohl rozum<G>ět?
\kr

\zs
Sbor kajícných mnichů jde krajinou v tichu
a pro všechnu lidskou pýchu má jen přezíravý smích.

Z prohraných válek se vojska domů vrací,
ač zbraně stále burácí a bitva zuří v~nich.
\ks

\zr
Sáro...
\kr

\zs
Vévoda v zámku čeká na balkóně,
až přivedou mu koně, pak mává na pozdrav.

Srdcová dáma má v každé ruce růže,
tak snadno pohřbít může sto urozených hlav.
\ks

\zr
Sáro...
\kr

\zs
Královnin šašek s pusou od povidel
sbírá zbytky jídel a myslí na útěk

a v podzemí skrytí slepí alchymisté
už objevili jistě proti povinnosti lék.
\ks

\zr
Sáro, Sáro, v noci se mi zdálo,
že tři andělé boží k nám přišli na oběd.

Sáro, Sáro, jak moc a nebo málo
ti chybí, abys mojí duši mohla rozumět?
\kr

\zs
Páv pod tvým oknem zpívá, sotva procit,
o tajemstvích noci ve tvých zahradách.

A já, potulný kejklíř, co svázali mu ruce,
teď hraju o tvé srdce a chci mít tě nadosah.
\ks

\zr
Sáro, Sáro, pomalu a líně
s hlavou na tvém klíně chci se probouzet.

Sáro, Sáro, Sáro, Sáro rosa padá ráno
a v poledne už možná bude jiný svět.
\kr

\zs
<Ami>Sáro, <Emi>Sáro, <F>vstávej, milá <C>Sáro,
<F>andělé k nám <Dmi>přišli na o<Cmaj>běd.
\ks

\kp
